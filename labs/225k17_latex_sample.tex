% 225k17_latex_sample.tex
% 
% This is a sample LaTeX source file with examples of LaTeX code similar to
% what is needed to type a CSC 225 assignment.
%
% There are numerous programs for editing and compiling LaTeX documents.
% The most basic way to produce a PDF from LaTeX source is the command line
% 'pdflatex' program. On the (linux) ECS 242 machines, the command would be
%	pdflatex 225k17_latex_sample.tex
%
% Note: A common compile error is caused by having the PDF file open while
% trying to compile a new version. In some cases, this results in the LaTeX
% compiler being unable to write to the file (so it gives a cryptic error).
% Make sure you close your PDF reader before recompiling.
%
% One good online reference for LaTeX commands is the Wikibook at
% 	http://en.wikibooks.org/wiki/LaTeX
%
% B. Bird - 05/17/2017


% If you don't care about background information and just want to type your
% assignment, scroll down to line 81 (to set the title and your name), then
% scroll to 97 to start writing.





% LaTeX is essentially a programming language: it has functions, variables
% and scoping rules. Most people just use it to generate formatted text,
% though.
%
% A '%' character begins a comment, which lasts until the end of a line.
% Whitespace is usually irrelevant (in most cases, one space has the same
% effect has several spaces or tabs).
% Commands begin with the \ character. The first command in a LaTeX document
% is usually the \documentclass command, which defines the type of document
% being created. We will use the 'artikel3' class, which is a good choice for
% general purpose use. Also good for general use is the 'article' class, which
% is similar to 'artikel3' but with different paragraph formatting (which tends
% to be better for written material but not as pleasant for assignments).
% Other classes include 'book', 'report', 'letter' and 'beamer' (for slides).
% The general structure of a LaTeX command is
%	\commandname{argument1}{argument2}{etc.}
% Some commands (including \documentclass) also have optional arguments, which
% can be omitted. The optional argument for \documentclass allows us to specify
% extra parameters about the document. We will specify letter-sized paper
% and a 12pt font.

\documentclass[letterpaper, 12pt]{artikel3}

% After the \documentclass command, most LaTeX documents also include several
% packages. The \usepackage command imports commands or specifications from
% an external library (similar to the import command in Java).

% We will only use some basic packages.

% Use the 'fullpage' package to alter the margins to fill the entire page.
% (By default, the artikel3 class uses very limited margins).

\usepackage{fullpage}

% These packages add a few useful math symbols.
\usepackage{amssymb}
\usepackage{amsmath}
\usepackage{mathtools}

% The verbatim package allows pre-formatted text (like code) to be included.
\usepackage{verbatim}

% The algorithmic package provides a somewhat convenient way to typeset
% algorithms.
\usepackage{algorithmic}

% Many LaTeX documents also contain definitions for custom commands
% in this part of the source file.


% Set the title, author and date
\title{CSC 225 LaTeX sample}
% The '\today' command inserts the current date
\date{ \today }
\author{Insert Name Here}

% After all of the desired packages have been included, the actual text
% of the document is written between the commands \begin{document}
% and \end{document}
%
\begin{document}

% The '\maketitle' command inserts the title/author/date information from
% above as a nicely formatted title.
% (You could also make the title manually using centering and font commands)
\maketitle

This is some ordinary text.
LaTeX ignores single newlines (so two adjacent lines appear as part of the same paragraph).

Leaving a blank line starts a new paragraph.

% To number the questions in the assignment, the \section* command can be
% used (even though it's intended primarily for chapter and section headings
% in reports).

\section*{Question 1}

% To typeset a mathematical formula, put it between $ characters.
% Try to avoid putting text inside of math delimiters (since it will be
% formatted strangely).

Here is some math: $n^2 + n - 1000 > 0$ for all $n > n_0$

% Exponents are created with ^ and subscripts are created with _
% Note that by default, only the first character after ^ or _ is
% used for an exponent or subscript.

Bad attempt at writing n to the power 123: $n^123$.

% To group text together, enclose it in curly brackets. This allows
% multi-character exponents:

Better attempt at writing n to the power 123: $n^{123}$.

% Square roots can be created with the \sqrt command.
% Everything inside the argument to \sqrt is placed under the radical.

Square root of 2 times n plus 1: $\sqrt{2n + 1}$

% To create fractions, use the \frac command.
% The first argument of \frac is the numerator, and the second argument
% is the denominator.

Here is a simple fraction: $\frac{1}{2}$

% Mathematical formulas can be nested:

Here is a less simple fraction: $\frac{n^2 + n + 100}{\sqrt{2}}$


% The 'regular' math typesetting using $ is designed to put math inline
% with text.

Using the definition of Big-O, we can prove that $n^2 \in O(n^3)$ and $5n^3 \in \Theta(n^3)$.

% Sometimes, the inline representation gets too crowded and looks ugly.

The limit of $\frac{n^2 + n + 100}{n^{\sqrt{n}}}$ converges to 0 as n goes to infinity.

% Enclosing math between \[ and \] typesets it in 'display mode', which centers
% it and ensures that everything is properly sized.

The limit of 
	\[\frac{n^2 + n + 100}{n^{\sqrt{n}}}\]
converges to 0 as n goes to infinity.

% Summations can be written with the \sum command
% The limits of the summation are specified using subscripts and exponents.

By induction, we can prove that
	\[\sum_{i = 0}^{n} 2^i = 2^{n+1} - 1\]
	
% Since curly brackets have a special meaning, you must use \{ and \} if you
% want them to appear in text.

If $A = \{1,2,3\}$ and $B = \{4,5,6\}$, then
	\[A \cap B = \emptyset\] 
	
% Many LaTeX commands (like \emptyset) are intuitively named.
% Some (like \cap for the set-intersection sign) are not.
% It is usually helpful to look up a reference sheet for LaTeX symbols
% online.

\section*{Question 2}
Sometimes, you may want to include raw text (including \$ and \{ characters) without
having to escape all the special characters.

% To add raw text, put it between a \begin{verbatim} and \end{verbatim}
% commands:
\begin{verbatim}
This is raw text.
Look at all these special characters: !@#$%^&*{}
LaTeX code in verbatim text is ignored: \begin{document} $n \leq n^2$ \end{document}
\end{verbatim}

\section*{Question 3}

Below is a proof of the identity
\[\sum_{i=0}^n i = \frac{n(n+1)}{2}\]
for all $n \geq 0$.

%To boldface text, use the \textbf command

\textbf{Basis}: When $n = 0$, $\sum_{i=0}^n i = 0$, and $\frac{n(n+1)}{2} = 0$, so the identity holds.

\textbf{Induction Hypothesis}: Suppose $\sum_{i=0}^n i = \frac{n(n+1)}{2}$ for some $n \geq 0$.

\textbf{Induction Step}:
		Consider $n+1$.
		% To produce a set of aligned equations (often helpful in proofs), use the align* environment
		% The align* environment produces three 'columns', delimited by & symbols
		% Usually, you only need two columns (left and right side of an equals sign).
		% In the proof below, the third column (which appears on the right side of the page)
		% is used for the text 'by the induction hypothesis'. 
		% The \text command is needed when displaying text inside of align*, since everything
		% inside align* is treated as math by default.
		% To end each line, use \\
		% If you don't use \\ at the end of a line, the LaTeX compiler may give an error.
		\begin{align*}
			\sum_{i=0}^{n+1} i & = (n+1) + \sum_{i=0}^n i\\
							   & = (n+1) + \frac{n(n+1)}{2} & \text{(By the induction hypothesis)}\\
							   & = \frac{2n + 2 + n^2 + n}{2}\\
							   & = \frac{n^2 + 3n + 2}{2}\\
							   & = \frac{(n+1)(n+2)}{2}
		\end{align*}
		Therefore, the identity holds for $n+1$, and by induction, the identity holds for all $n \geq 0$.

\section*{Question 4}

Here is some pseudocode for an algorithm which computes the sum
	\[\sum_{i=0}^n i\]
and returns the computed value.
\begin{algorithmic}
	\STATE $x \gets 0$
	\FOR{$i \gets 0, 1, 2, \ldots, n$}
		\STATE $x \gets x + i$
	\ENDFOR
	\RETURN $x$
\end{algorithmic}

The pseudocode can also be written with a while loop.
\begin{algorithmic}
	\STATE $x \gets 0$
	\IF{$n = 0$}
		\RETURN 0
	\ENDIF
	\STATE $i \gets 0$
	\WHILE{$i \leq n$}
		\STATE $x \gets x + i$
		\STATE $i \gets i + 1$
	\ENDWHILE
	\RETURN $x$
\end{algorithmic}





\end{document}
